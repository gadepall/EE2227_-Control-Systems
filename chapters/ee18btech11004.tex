%\documentclass{beamer}
%
% Choose how your presentation looks.
%
% For more themes, color themes and font themes, see:
% http://deic.uab.es/~iblanes/beamer_gallery/index_by_theme.html
%
%\mode<presentation>
%{
%  \usetheme{NYU}      % or try Darmstadt, Madrid, Warsaw, ...
%  \usecolortheme{default} % or try albatross, beaver, crane, ...
%  \usefonttheme{default}  % or try serif, structurebold, ...
%  \setbeamertemplate{navigation symbols}{}
% \setbeamertemplate{caption}[numbered]
%} 

%\usepackage[english]{babel}
%\usepackage[T1]{fontenc}
%\usepackage[utf8x]{inputenc}
%\usepackage{hyperref}
%\begin{document}
%\title[EE2227]{CONTROL SYSTEMS}
%subtitle{HOMEWORK-1}
%\author{BHUKYA SIDDHU\\ EE18BTECH11004}
%\institute{IITH}
%\date{\today}



%\begin{frame}
%  \titlepage
%\end{frame}
%\item \textbf{Question 33}
\begin{enumerate}[label=\thesection.\arabic*.,ref=\thesection.\theenumi]
\numberwithin{equation}{enumi}
\item Let the state-space representation of an LTI system be.
\begin{align*}
 \dot{x(t)}=Ax(t)+Bu(t) \\
 y(t)=Cx(t)+Du(t)
\end{align*}
A,B,C are matrices, D is scalar, u(t) is input to the system and y(t) is output to the system. let
\begin{equation}
 b1 =\begin{vmatrix}
  0&0&1\\
 \end{vmatrix}
\end{equation}
\begin{align}
b1^T=B
\end{align}
and D=0. Find A and C.
\begin{align}
H(s)=\frac{1}{s^3+3s^2+2s+1}
\end{align}
\solution
\\ \textbf{FINDING TRANSFER FUNCTION}
\\so
\begin{align*}
 \dot{X(t)}=AX(t)+BU(t) \\
 Y(t)=CX(t)+DU(t)
\end{align*}    
\\by applying laplace transforms on both sides of equation 1
we get
\\S.X(S)-X(0)=A.X(S)+B.U(S)
\\S.X(S)-A.X(S)=B.U(S)+X(0)
\\(SI-A)X(S)=X(0)+B.U(S)
\\$X(S)=X(0)([SI-A])^{-1}+ B([SI-A])^{-1}U(S)$
\\Laplace transform of equation 2 and sub X(s) 
\\Y(S)=C.X(S)+D.U(S)
\\$Y(S)=C.[X(0)([SI-A])^{-1 }+ B([SI-A])^{-1}U(S)]+DU(S)$
\\If X(0)=0
\\then $Y(S)=C[B([SI-A])^{-1}U(S)]+DU(S)$
\\$\frac{Y(S)}{U(S)}=C[B([SI-A])^{-1}]+D=H(S)$
\\
\\ As we know that
\\ $Y(s)=H(s) \times U(s)= (\frac{1}{s^3+3s^2+2s+1}) \times U(s) $
\\
\\ $H(s)=\frac{Y(s)}{U(s)}=(\frac{x_{1}(s)}{U(s)})\times \frac{Y(s)}{x_{1}(s)}$
\\
\\let   $x_{1}(s)=\frac{U(s)}{denominator}$
\\
\\$Y(s)=x_{1}(s)\times numerotor$
\begin{align}
s^3x_{1}(s)+3s^2x_{1}(s)+2sx_{1}(s)+x_{1}(s)=U(S)
\end{align}
\\ Taking inverse laplace transform we get
\\$\dddot{x_{1}(t)}+\ddot{x_{1}(t)}+\dot{x_{1}(t)}+x_{1}(t)=U(t)$
\\$\dot{x_{1}}=x_{2}$
\\$\ddot{x_{1}}=\dot{x_{2}}=x_{3}$
\\$\dddot{x_{1}}=\ddot{x_{2}}=\dot{x_{3}}$
\\ so equation 1.1.4 can be written as
\\
\begin{gather}
\begin{bmatrix}
sx_{1}(s)\\
s^2x_{1}(s)\\
s^3x_{1}(s)
\end{bmatrix}
=
\begin{bmatrix}
0&1&0\\
0&0&1\\
-1&-2&-3
\end{bmatrix}\times \begin{bmatrix}
x_{1}(s)\\
sx_{1}(s)\\
s^2x{1}(s)
\end{bmatrix}
+
\begin{bmatrix}
0\\
0\\
1
\end{bmatrix} \times U
\end{gather}
taking inverse laplace transform
\begin{gather}
\begin{bmatrix}
\dot{x_{1}}\\
\dot{x_{2}}\\
\dot{x_{3}}
\end{bmatrix}
=
\begin{bmatrix}
0&1&0\\
0&0&1\\
-1&-2&-3
\end{bmatrix}\times \begin{bmatrix}
x_{1}\\
x_{2}\\
x_{3}
\end{bmatrix}
+
\begin{bmatrix}
0\\
0\\
1
\end{bmatrix} \times U
\end{gather}
therfore
\begin{equation}
A=\begin{bmatrix}
0&1&0\\
0&0&1\\
-1&-2&-3
\end{bmatrix}
\end{equation}
\\
Since $ Y(s)=x_{1}(s)\times numerator$
\\therefore $ Y(s)=x_{1}(s) $
\begin{gather}
\\Y=
\begin{bmatrix}
1&0&0
\end{bmatrix}\times \begin{bmatrix}
x_{1}(s)\\
sx_{1}(s)\\
s^2x_{1}(s)
\end{bmatrix} 
\end{gather}
taking inverse laplace transform
\begin{gather}
\\Y=
\begin{bmatrix}
1&0&0
\end{bmatrix}\times \begin{bmatrix}
x_{1}\\
x_{2}\\
x_{3}
\end{bmatrix} 
\end{gather}
\begin{equation}
C=\begin{bmatrix}
1&0&0
\end{bmatrix}
\end{equation}
%\end{document}

%\end{document}
\end{enumerate}

